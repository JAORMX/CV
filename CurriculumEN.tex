\documentclass[english,10pt,letterpaper]{article}
\usepackage[TextAligned]{currvita}

\usepackage[cm]{fullpage}
\usepackage[absolute,overlay]{textpos}

\usepackage[utf8]{inputenc}
\usepackage[english]{babel}

\usepackage{url}
\usepackage[colorlinks,urlcolor=blue]{hyperref}

\usepackage[pdftex]{color,graphicx}

\usepackage{palatino} % Palatino, Helevetica, Courier

\begin{document}
\centering
\begin{cv}{Juan Antonio Osorio-Robles}

	\begin{table}[h]
		\begin{tabular}{@{} l l p{0.5cm} l r}
			{\bf Contact}	&\\
			email & \href{mailto:jaosorior@gmail.com}{\tt jaosorior@gmail.com}\\
		\end{tabular}
	\end{table}

	\begin{cvlist}{Introduction}
        \item[\textsc{Summary}]
            I'm a software engineer passionate about learning new technologies
            and building software the best and most skillful way that I can.
	\end{cvlist}

	\begin{cvlist}{Webpages \& Code Repositories}
        \item [\textsc{Github}]
            \url{http://github.com/JAORMX}

        \item [\textsc{Blog}]
            \url{https://jaosorior.dev/}
	\end{cvlist}

	\begin{cvlist}{Professional Experience}
		\item [March 2019 - present]
            \textbf{Principal Software Engineer @ Red Hat} (Finland)\\
            I lead a team with a focus on improving security and ensuring that
            OpenShift \& RHCoreOS meet standards required in regulated
            industries. \\

            Here, aside from assessing security controls and making sure the
            products can fulfil them. We developed the
            \href{https://github.com/openshift/compliance-operator}{compliance-operator},
            which helps folks both evaluate and get closer to compliance with
            different security standards in a declarative and cloud-native
            manner. We also developed the 
            \href{https://github.com/openshift/file-integrity-operator}{file-integrity-operator},
            which ensures files are not modified without the administrators'
            consent, helping secure the deployment.



		\item [September 2015 - March 2019]
            \textbf{Senior Software Engineer @ Red Hat} (Finland)\\
            I'm part of the OpenStack-focused Identity/Security team in Red
            Hat. I've driven efforts to improve the security of the deployment
            overall, such as TLS enablement, policy enhancements, IPSec,
            FreeIPA integration; all following an upstream-first approach.

            \url{http://stackalytics.com/?metric=commits&release=all&user_id=jaosorior&company=red\%20hat}

		\item [April 2014 - September 2015]
            \textbf{Senior Software Developer @ Ericsson} (Finland)\\
            I drove \textbf{OpenStack} open source security-related
            contributions. I also actively took part of the \textbf{OPNFV}
            community, where we started the \textbf{Inspector} project to
            enhance the auditability of OpenStack. \\

            \url{http://stackalytics.com/?metric=commits&release=all&user_id=jaosorior&company=ericsson}

            Before this I was part of an integration task force, with the
            aim of creating a sustainable development environment for
            the engineers to build on.\\

            For a while I was also part of the Cloud Security team where
            I did numerous tasks, such as implement security related
            features; harden and test the security in our cloud solution,
            among other activities.

		\item [June 2012 - April 2014]
            \textbf{Software Developer @ Ericsson} (Finland)\\
            I've been taken part in the development of high performance
            multimedia systems; coded scripts to aid in version control
            systems migration (migrated from a legacy VCS to \textbf{Git});
            took part in in a project to develop a visualization of the
            development process within the division; amongst other things.\\

            Aside from that, every once in a while I gave guidance
            regarding open-source processes (licensing and implications).

		\item [July 2011 - December 2011]
            \textbf{Research assistant/Developer @ NIC Mexico} (Monterrey,
            N.L., M\'{e}xico)\\
            Developer lead of a Linux implementation for the \textbf{NAT64}
            mechanism (based on the RFC6146), which had the purpose of
            enabling communication between IPv6 and IPv4 hosts. It was
            implemented as a \textbf{kernel module}, written in \textbf{C},
            working with \textbf{Netfilter}. For user-space control, an
            iptables module was also developed. The implementation is
            open-source:
            \url{http://jool.mx/} \&\\
            \url{https://github.com/NICMx/NAT64}

	\end{cvlist}

    %\begin{cvlist}{Other Project(s)}
	%	\item [January 2012 - May 2012]
    %        \textbf{Operating Systems Project}(Aalto University, Finland)\\
    %        Successfully extended an incomplete operating system; implemented
    %        the scheduler, system calls, file system, network driver,
    %        concurrency manager, memory management, a shell, among other
    %        things; The target hardware was based on the MIPS architecture.
	%\end{cvlist}

	\begin{cvlist}{Technical Skills}
			\item [\textsc{Languages}]
                Golang, Python, bash, C, C++, Ruby, Java, Javascript
			\item [\textsc{Technologies}]
                OpenShift, Kubernetes, OpenStack, Ansible, Puppet,
                Docker/Podman, libvirt, GStreamer
			\item [\textsc{Development Tools}]
                Git, Gerrit, Subversion, Mercurial, GNU Make, CMake
			\item [\textsc{SW Development Methods}]
				Fond of Continuous Integration for Software.\\
                Test-Driven Development\\
                Agile Methodologies (Kanban, \& Scrum)
			\item [\textsc{Operating Systems}]
                Linux (CoreOS, RHEL, CentOS, Archlinux and Fedora)
	\end{cvlist}

    \begin{cvlist}{Books \& Publications}
		\item []
            (May 2020) \href{https://www.amazon.com/dp/B087YSL9L2}{OpenShift Security Guide}
	\end{cvlist}

    \begin{cvlist}{Open Source Community Positions}
		\item []
            I'm a core-developer in the \textbf{Barbican} project (Secret
            Storage As a Service) and \textbf{TripleO} project (OpenStack
            over Openstack - a cloud deployment tool) in the OpenStack
            community. I aslo served as the Project Technical Lead for the
            TripleO project, and lead the TripleO Security Squad.
	\end{cvlist}

	\begin{cvlist}{Education}
		\item [January 2012 - December 2012]
			Aalto University, School of Science, Helsinki, Finland\\
			Exchange student. Computer Science Department in the School of
            Science\\
			\href{www.aalto.fi/en}{www.aalto.fi/en}

		\item	[August 2008-- August 2013]
			ITESM, Campus Monterrey. Monterrey N.L., M\'{e}xico\\
			\emph{BCT - B.S. Computer Science and Technology (4.5 years)}\\
			MI - International Modality(
            \url{http://www.mty.itesm.mx/rectoria/dda/dacmi})\\
			Education grade point average: \emph{92/100}
	\end{cvlist}

	\begin{cvlist}{Languages}
		\item [\textsc {Spanish}]
				Native
		\item [\textsc {English}]
				Working language
		\item [\textsc {Finnish}]
				Basic
		\end{cvlist}

	%\begin{cvlist}{Other Activities}
	%	\item [2010 - 2012] GULTec - ITESM Linux User Group.\\
	%		I organized and helped organize technical workshops, which were
    %        given for free either by me, other members of the group, or by
    %        professionals. I also was part of the organization of the event
    %        called ``Interunivesitario de Software Libre''. There, I gave a
    %        presentation about Open-Source Compliance.
    %\end{cvlist}

\end{cv}

\end{document}
