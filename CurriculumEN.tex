\documentclass[spanish,10pt,letterpaper]{article}
% \usepackage[LabelsAligned]{currvita}		% compact
\usepackage[TextAligned]{currvita}

\usepackage[cm]{fullpage}
\usepackage[absolute,overlay]{textpos}

\usepackage[utf8]{inputenc}
\usepackage[spanish,english]{babel}

\usepackage{url}
\usepackage[colorlinks,urlcolor=blue]{hyperref}

\usepackage[pdftex]{color,graphicx}

%\usepackage{newcent} % New Century Schoolbook, Avant Garde, Courier
\usepackage{palatino} % Palatino, Helevetica, Courier
%\usepackage{bookman}
%\usepackage{lmodern}

% begin CV
\begin{document}
\centering
\begin{cv}{Juan Antonio Osorio-Robles}

	\begin{table}[h]
		\begin{tabular}{@{} l l p{0.5cm} l r}
			{\bf Personal Information}	&	&	&	{\bf Contact}	&\\
			Date of birth &	January 8, 1990 (25 years old)	&	&
			email & \href{mailto:jaosorior@gmail.com}{\tt jaosorior@gmail.com}\\
		\end{tabular}
	\end{table}

	\begin{cvlist}{Introduction}
        \item[\textsc{Summary}]
            I'm a Mexican software developer aiming to achieve mastery in
            Software Craftsmanship. Meaning I take time and pride in refining
            methods and practices in order to achieve this. I'm very passionate
            about learning new technologies and building software the best and
            most skillful way that I can.
        \item[\textsc{Main assets}]
            Focus on maintainable, elegant and clean code.\\
            I will go as deep as I need to go to fix a problem.\\
            Comfortable in big code bases.\\
            I will not only focus on the technical side, I will also try to get
            a good grasp of the big picture and even the strategy behind it.
	\end{cvlist}


	\begin{cvlist}{Technical Skills}
			\item [\textsc{Languages}]
                Python (main language), C (professional experience), Java
                (can review code but need to time remember), SQL, Javascript,
                Scheme, PHP, bash scripting (bread of every day)
			\item [\textsc{Technologies}]
                OpenStack (I'm an Active Technical Contributor), Docker,
                KVM (Basic experience to set my environment for OpenStack),
                Puppet (basic experience), GStreamer
			\item [\textsc{Development Tools}]
                Git (proficient), Vim (main editor), Subversion, Mercurial,
                Eclipse (I'll use if needed)
			\item [\textsc {SW Development Methods}]
				Fond of the Continuous Integration for Software.\\
                TDD: I've given small workshops at work about it.\\
                I've also worked with Agile Methodologies (Kanban, Extreme
                Programming \& Scrum)
			\item [\textsc {Operating Systems}]
                Mostly have worked with Linux (Archlinux as main distro, but
                have also used Ubuntu, Debian and to a smaller extent
                Fedora).\\
                I'm eager to learn about other distros and even OS's (specially
                open ones)
	\end{cvlist}

	\begin{cvlist}{Professional Experience}
		\item [April 2014 - Present]
            \textbf{Senior Software Developer @ Ericsson} (Finland)\\
            My main task is to contribute directly to the
            \textbf{OpenStack} project as part of a team dedicated to
            open-source entirely; Here I actively interact with the
            community, review commits and push code to the projects. I'm a
            \textbf{core-developer} in the \textbf{Barbican} project in
            OpenStack and also an active contributor to the \textbf{OPNFV}
            initiative, where I form part of the Security Group and forming
            the audit-oriented \textbf{Inspector} project there.\\

            \href{http://stackalytics.com/?metric=commits&user_id=juan-osorio-robles}
            {\url{http://stackalytics.com/?metric=commits&user_id=juan-osorio-robles}}

            Before this I was part of an integration task force, with the
            aim of creating a sustainable development environment for
            developers to build on. The environment was based on
            \textbf{Docker} in order to ensure the uniformity of
            environments and to create quick and consistent builds.\\

            For a while I was also part of the Cloud Security team where
            I did numerous tasks, such as implement security related
            features; harden and test the security in our cloud solution,
            among other activities.

		\item [June 2012 - April 2014]
            \textbf{Software Developer @ Ericsson} (Finland)\\
            Worked in multimedia-based applications, as a developer in
            Ericsson's Media Gateway. Here, The type and nature of the
            tasks I performed were diverse: I've been taken part in the
            development of high performance multimedia systems; coded
            scripts to aid in version control systems migration (passed
            from a legacy VCS to \textbf{Git}); took part in developing a
            visualization of the development process within the division;
            amongst other things.\\

            Aside from that, every once in a while I gave guidance
            regarding Open Source processes (licensing and implications).

		\item [July 2011 - December 2011]
            \textbf{Research assistant/Developer @ NIC Mexico} (Monterrey,
            N.L., M\'{e}xico)\\
            Developer lead of a Linux implementation for the \textbf{NAT64}
            mechanism (based on the RFC6146), which had the purpose of
            enabling communication between IPv6 and IPv4 hosts. It was
            implemented as a \textbf{kernel module}, written in \textbf{C},
            working with \textbf{Netfilter}, which would be the kernel-space
            component. For user-space control, an iptables module was also
            being developed. The implementation is open source:
            \href{http://jool.mx/}{http://jool.mx/} \&
            \href{https://github.com/NICMx/NAT64}{https://github.com/NICMx/NAT64}.

		\item [May 2010 - May 2011]
            \textbf{Research assistant @ ITESM} (Monterrey, N.L., M\'{e}xico)\\
            I developed an application (Gadget) for the \textbf{Google Wave}
            platform to collaboratively draw ER diagrams. The project used web
            technologies such as \textbf{HTML5} and the Wave platform's API,
            which is based in \textbf{Java}. Also, I also contributing to the
            writing of the research paper, but mostly my role was technical.\\

            I also contributed to the Open Source project Wave In a Box,
            studied its infrastructure and adapted code to fit the
            University's project. I was responsible for administrating an
            Ubuntu Server with a co-worker inside Campus for the project's
            testing purposes.

		\item [Summer 2010]
            \textbf{Freelance}(Monterrey, N.L., M\'{e}xico)\\
			A co-worker and I developed an information system to administer an
            apartment complex. It managed the tentants' contracts and payments,
            the apartments' availability and the printing of contracts,
            receipts and bills.\\

            The system is being used by the landord even today.

	\end{cvlist}

    \begin{cvlist}{Other Project(s)}
		\item [January 2012 - May 2012]
            \textbf{Operating Systems Project}(Aalto University, Finland)\\
            With a team consisting of me and two classmates, we successfully
            extended an incomplete operating system, implementing the scheduler,
            system calls, file system, network driver, concurrency manager,
            memory management, a shell, among other things, for a MIPS
            processor.
	\end{cvlist}

	\begin{cvlist}{Education}
		\item [January 2012 - December 2012]
			Aalto University, School of Science, Helsinki, Finland\\
			Exchange student. Computer Science Department in the School of
            Science\\
			\href{www.aalto.fi/en}{www.aalto.fi/en}

		\item	[August 2008-- August 2013]
			ITESM, Campus Monterrey. Monterrey N.L., M\'{e}xico\\
			\emph{BCT - B.S. Computer Science and Technology (4.5 years)}\\
			MI - International Modality(
            \href{http://www.mty.itesm.mx/rectoria/dda/dacmi}{http://www.mty.itesm.mx/rectoria/dda/dacmi})\\
			Education grade point average: \emph{92/100}
	\end{cvlist}

	\begin{cvlist}{Awards \& Certifications}
		\item [October 2011] Second Place winners of the ITESM's ACM Hackathon.
            As a team of 4 members, we developed an application to control a
            Linux computer from twitter.(Monterrey, N.L., M\'{e}xico)
		\item [August - November 2011] Winners of the ITESM's Business Model
            presentation fair and the Business Model Challenge. Developed the
            business model for a company called BIOFASE (team of 5) that sells
            bioplastic-based products. The project was developed for an
            entrepreneurship class and competed with other projects from all
            the entrepreneurship classes that semester.
            (Monterrey, N.L., M\'{e}xico).
		\item [October-December 2011] Successfully finished the Introduction to
            Artificial Intelligence with a 90.2/100 grade. \\
		    \href{http://ai-class.com}{http://ai-class.com}. (Online course
            given in partnership with Stanford Engineering)
	\end{cvlist}


	\begin{cvlist}{Languages}
		\item [\textsc {Spanish}]
				Native
		\item [\textsc {English}]
				TOEFL IBT: \emph{103/120}
		\item [\textsc {German}]
				Intermediate, 2 years
		\end{cvlist}

	\begin{cvlist}{Other Activities}
		\item [2010 - 2012] GULTec - ITESM Linux User Group.\\
			I organized and helped organize technical workshops, which were
            given for free either by me, other members of the group, or by
            professionals. I also was part of the organization of the event
            called "Interunivesitario de Software Libre 2011", which took place
            in various universities at the same time and was transmitted via
            Ustream. There, I gave a presentation about Open Source Compliance.
            I had contact with some companies and gave announcements to the
            group members about the vacancies available.
	\end{cvlist}

	\begin{cvlist}{Webpages \& Code Repositories}
		\item [Github] \href{http://github.com/JAORMX}{http://github.com/JAORMX}
	\end{cvlist}

\end{cv}

\end{document}
