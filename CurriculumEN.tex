\documentclass[english,10pt,letterpaper]{article}
\usepackage[TextAligned]{currvita}

\usepackage[cm]{fullpage}
\usepackage[absolute,overlay]{textpos}

\usepackage[utf8]{inputenc}
\usepackage[english]{babel}

\usepackage{url}
\usepackage[colorlinks,urlcolor=blue]{hyperref}

\usepackage[pdftex]{color,graphicx}

\usepackage{palatino} % Palatino, Helevetica, Courier

\begin{document}
\centering
\begin{cv}{Juan Antonio Osorio-Robles}

	\begin{table}[h]
		\begin{tabular}{@{} l l p{0.5cm} l r}
			{\bf Personal Information}	&	&	&	{\bf Contact}	&\\
			Date of birth &	January 8, 1990	&	&
			email & \href{mailto:jaosorior@gmail.com}{\tt jaosorior@gmail.com}\\
		\end{tabular}
	\end{table}

	\begin{cvlist}{Introduction}
        \item[\textsc{Summary}]
            I'm a software developer passionate about learning new technologies
            and building software the best and most skillful way that I can.
	\end{cvlist}

	\begin{cvlist}{Webpages \& Code Repositories}
        \item [\textsc{Github}]
            \href{http://github.com/JAORMX}{http://github.com/JAORMX}
	\end{cvlist}

    \begin{cvlist}{Open Source Community Positions}
		\item [OpenStack]
            I'm a core-developer in the \textbf{Barbican} project (Secret
            Storage As a Service) and \textbf{TripleO} project (OpenStack
            over Openstack - a cloud deployment tool) in the OpenStack
            community. I'm also leading the Tripleo Security Squad. I'm also
            one of the maintainers for the \textbf{puppet-certmonger} puppet
            manifest.
	\end{cvlist}

	\begin{cvlist}{Professional Experience}
		\item [September 2015 - present]
            \textbf{Senior Software Developer @ Red Hat} (Finland)\\
            I'm part of the OpenStack-focused Identity/Security team in Red
            Hat. I've driven efforts to improve the security of the deployment
            overall, such as TLS enablement, IPSec, FreeIPA integration; all
            following an upstream-first approach.

            \href{http://stackalytics.com/?metric=commits&release=all&user_id=jaosorior&company=red\%20hat}
            {\url{http://stackalytics.com/?metric=commits&release=all&user_id=jaosorior&company=red\%20hat}}

		\item [April 2014 - September 2015]
            \textbf{Senior Software Developer @ Ericsson} (Finland)\\
            I drove \textbf{OpenStack} open source security-related
            contributions. I also actively took part of the \textbf{OPNFV}
            community, where we started the \textbf{Inspector} project to
            enhance the auditability of OpenStack. \\

            \href{http://stackalytics.com/?metric=commits&release=all&user_id=jaosorior&company=ericsson}
            {\url{http://stackalytics.com/?metric=commits&release=all&user_id=jaosorior&company=ericsson}}

            Before this I was part of an integration task force, with the
            aim of creating a sustainable development environment for
            the engineers to build on.\\

            For a while I was also part of the Cloud Security team where
            I did numerous tasks, such as implement security related
            features; harden and test the security in our cloud solution,
            among other activities.

		\item [June 2012 - April 2014]
            \textbf{Software Developer @ Ericsson} (Finland)\\
            I've been taken part in the development of high performance
            multimedia systems; coded scripts to aid in version control
            systems migration (migrated from a legacy VCS to \textbf{Git});
            took part in in a project to develop a visualization of the
            development process within the division; amongst other things.\\

            Aside from that, every once in a while I gave guidance
            regarding open-source processes (licensing and implications).

		\item [July 2011 - December 2011]
            \textbf{Research assistant/Developer @ NIC Mexico} (Monterrey,
            N.L., M\'{e}xico)\\
            Developer lead of a Linux implementation for the \textbf{NAT64}
            mechanism (based on the RFC6146), which had the purpose of
            enabling communication between IPv6 and IPv4 hosts. It was
            implemented as a \textbf{kernel module}, written in \textbf{C},
            working with \textbf{Netfilter}. For user-space control, an
            iptables module was also developed. The implementation is
            open-source:
            \href{http://jool.mx/}{http://jool.mx/} \&\\
            \href{https://github.com/NICMx/NAT64}{https://github.com/NICMx/NAT64}.

		\item [May 2010 - May 2011]
            \textbf{Research assistant @ ITESM} (Monterrey, N.L., M\'{e}xico)\\
            I developed an application (Gadget) for the \textbf{Google Wave}
            platform to collaboratively draw ER diagrams. The project used web
            technologies such as \textbf{HTML5} and the Wave platform's API,
            which is based in \textbf{Java}. Also, I contributed to the
            writing of the research paper, but mostly my role was technical.

	\end{cvlist}

    \begin{cvlist}{Other Project(s)}
		\item [January 2012 - May 2012]
            \textbf{Operating Systems Project}(Aalto University, Finland)\\
            Successfully extended an incomplete operating system; implemented
            the scheduler, system calls, file system, network driver,
            concurrency manager, memory management, a shell, among other
            things; The target hardware was based on the MIPS architecture.
	\end{cvlist}

	\begin{cvlist}{Technical Skills}
			\item [\textsc{Languages}]
                Python (main language), C++, C, Go, bash scripting, Ruby, Java,
                SQL, Javascript
			\item [\textsc{Technologies}]
                OpenStack, Ansible, OpenShift, Puppet, Docker, libvirt, GStreamer
			\item [\textsc{Development Tools}]
                Git, Gerrit, Subversion, Mercurial, GNU Make, CMake
			\item [\textsc{SW Development Methods}]
				Fond of Continuous Integration for Software.\\
                Test-Driven Development\\
                Agile Methodologies (Kanban, Extreme Programming \& Scrum)
			\item [\textsc{Operating Systems}]
                Linux (I've used and administered CentOS, Archlinux and Fedora
                mainly, but have also used Ubuntu and Debian to some extent).
	\end{cvlist}

	\begin{cvlist}{Education}
		\item [January 2012 - December 2012]
			Aalto University, School of Science, Helsinki, Finland\\
			Exchange student. Computer Science Department in the School of
            Science\\
			\href{www.aalto.fi/en}{www.aalto.fi/en}

		\item	[August 2008-- August 2013]
			ITESM, Campus Monterrey. Monterrey N.L., M\'{e}xico\\
			\emph{BCT - B.S. Computer Science and Technology (4.5 years)}\\
			MI - International Modality(
            \href{http://www.mty.itesm.mx/rectoria/dda/dacmi}
            {http://www.mty.itesm.mx/rectoria/dda/dacmi})\\
			Education grade point average: \emph{92/100}
	\end{cvlist}

	\begin{cvlist}{Languages}
		\item [\textsc {Spanish}]
				Native
		\item [\textsc {English}]
				TOEFL IBT: \emph{103/120} - working language
		\item [\textsc {Finnish}]
				Basic
		\end{cvlist}

	\begin{cvlist}{Other Activities}
		\item [2010 - 2012] GULTec - ITESM Linux User Group.\\
			I organized and helped organize technical workshops, which were
            given for free either by me, other members of the group, or by
            professionals. I also was part of the organization of the event
            called ``Interunivesitario de Software Libre''. There, I gave a
            presentation about Open-Source Compliance.
    \end{cvlist}

\end{cv}

\end{document}
