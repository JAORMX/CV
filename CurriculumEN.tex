% TODO: Agregar actividades de liderazgo.
\documentclass[spanish,10pt,letterpaper]{article}
% \usepackage[LabelsAligned]{currvita}		% versión compacta
\usepackage[TextAligned]{currvita}
	
\usepackage[cm]{fullpage}			% márgenes de 1"
\usepackage[absolute,overlay]{textpos}		% posicionamiento absoluto

\usepackage[utf8]{inputenc}
\usepackage[spanish,english]{babel}

\usepackage{url}
\usepackage[colorlinks,urlcolor=blue]{hyperref}

\usepackage[pdftex]{color,graphicx}

%\usepackage{newcent} % New Century Schoolbook, Avant Garde, Courier 
\usepackage{palatino} % Palatino, Helevetica, Courier 
%\usepackage{bookman}
%\usepackage{lmodern}

% begin of CV
\begin{document}
\centering
\begin{cv}{Juan Antonio Osorio-Robles}

% Para insertar una imagen:
%	\begin{textblock}{0}(13,2)
%		\includegraphics[height=2.6cm]{foto.jpg}
%	\end{textblock}

	\begin{table}[h]
		\begin{tabular}{@{} l l p{0.5cm} l r}
			{\bf Personal Information}	&	&	&	{\bf Contact}	&	\\
			Date of birth &	January 8, 1990 (21 years old)	&	&
			email	&	\href{mailto:jaosorior@gmail.com}	{\tt jaosorior@gmail.com}	\\
			%Mobile &	{\tt +52~(1)~811.583.7910}	\\
			Mobile &	{\tt 0458507886}
		\end{tabular}
	\end{table}

	\begin{cvlist}{Introduction}
		\item[	]
			I'm a very patient and enthusiastic person that loves to learn. Even though I like to give problems a thorough analysis, I also enjoy team work and try to promote a relaxed work environment. Also, I love challenges and am very good handling stress, but I'm not afraid to ask for help when needed.
	\end{cvlist}

	
	\begin{cvlist}{Technical Skills}
			\item [\textsc{Programming Languages}]
				C(3 years), Java(3 years), Scheme(1.5 years), Python(6 months), SQL(2 years), Javascript(2 years), PHP(2.5 years), bash scripting (2 years)
			\item [\textsc{Development Tools}]
				Git, Subversion, Mercurial, Vim, Eclipse, gdb
			\item [\textsc {Services and Technologies}]
				Apache, Ajax, GTK+, JQuery, Netfilter, MySQL, Wave Gadgets \& Robots
			\item [\textsc {SW Development Methods}]
				Agile Methodologies (Extreme Programming, Scrum), with Object Oriented Design (UML)
			\item [\textsc {Markup Languages}]
				HTML, XML,
				\LaTeX
			\item [\textsc {Operating Systems}]
				Linux (good and enthusiast), I'm eager to learn about other OS's (specially open ones)
	\end{cvlist}

	\begin{cvlist}{Education}
%	\begin{cvlist}{Historia académica}
		\item [January 2012 - December 2012]
			Aalto University, School of Science, Helsinki, Finland
			\\
			Exchange student. Computer Science Department in the School of Science
			\\
			\href{www.aalto.fi/en}{www.aalto.fi/en}

		\item	[August 2008-- to date]
			ITESM, Campus Monterrey. Monterrey N.L., M\'{e}xico
			\\
			\emph{BCT - B.S. Computer Science and Technology (4.5 years)}
			\\
			MI - International Modality(\href{http://www.mty.itesm.mx/rectoria/dda/dacmi}{http://www.mty.itesm.mx/rectoria/dda/dacmi})
			\\
			Education grade point average: \emph{92/100}
	\end{cvlist}

	\begin{cvlist}{Professional Experience}
		\item [July 2011 - December 2011]
			Research project in ITESM for NIC Mexico (Monterrey, N.L., M\'{e}xico)
			\\ \emph{Architecture design, programming (development)}
			\\ Architect and developer lead of a Linux implementation for the NAT64 mechanism (based on the RFC6146), which had the purpose of enabling communication between IPv6 and IPv4 hosts. It was implemented as a kernel module, written in C, working with Netfilter, which would be the kernel-space component. For user-space control, an iptables module was also being developed. The implementation is open source under the GPLv3 license.
		\item [May 2010 - May 2011]
			Research assistant in ITESM (Monterrey, N.L., M\'{e}xico)
			\\ \emph{Architecture design, programming (development), user interface and testing}
			\\ I developed an application (Gadget) for the Google Wave platform to collaboratively draw ER diagrams. The project used web technologies such as HTML5 and the Wave platform's API. Also, I contributed to the research document and designed the states (information) that the robots would be sharing and updating with the Gadget. I also contributed to the Open Source project Wave In a Box, studied its infrastructure and adapted code to fit the University's project. I was responsable for administrating an Ubuntu Server with a co-worker inside Campus for the project's testing purposes.
		\item [Summer 2010]
			Administrative information system for the apartment complex ``Departamentos Coahuila'' (Monterrey, N.L., M\'{e}xico)
			\\ \emph{Programming (development), design and security}
			\\ A co-worker and I developed an information system to administer an apartment complex. It managed the tentants' contracts and payments, the apartments' availability and the printing of contracts, receipts and bills.
			
	\end{cvlist}

	\begin{cvlist}{Awards \& Certifications}
		\item [October 2011] Second Place winners of the ITESM's ACM Hackathon. As a team of 4 members, we developed an application to control a Linux computer from twitter.(Monterrey, N.L., M\'{e}xico)
		\item [August - November 2011] Winners of the ITESM's Business Model presentation fair and the Business Model Challenge. Developed the business model for a company called BIOFASE (team of 5) that sells bioplastic-based products. The project was developed for an entrepreneurship class and competed with other projects from all the entrepreneurship classes that semester. (Monterrey, N.L., M\'{e}xico).
		\item [October-December 2011] Successfully finished the Introduction to Artificial Intelligence with a 90.2/100 grade. \\
		\href{http://ai-class.com}{http://ai-class.com}.(Online course given in partnership with Stanford Engineering)
	\end{cvlist}


	\begin{cvlist}{Languages}
		\item [\textsc {Spanish}]
				Native
		\item [\textsc {English}]
				TOEFL IBT: \emph{103/120}
		\item [\textsc {German}]
				Intermediate, 2 years
		\end{cvlist}

	\begin{cvlist}{Interests}
		\item [\textsc {Technology}]
				Distributed computing, Linux programming and internals, operating systems internals, Open Source
		\item [  ] Linux user since 2008
		\item [  ] Open Source enthusiast
		\item [\textsc {Computer Science}]
				Intelligent systems, machine learning, programming languages
		\item [\textsc {Hobbies}]
				Guitar, philosophy, music, literature, films
	\end{cvlist}

	\begin{cvlist}{Other Activities}
		\item [2008-2010] SAITC - Student group for the Computer Technologies Major.\\
			I participated in the various events the Student group organized, notably, the Format Fest, Which is the Student group's biggest event. The event is held each semester and consists of a whole weekend fixing computers of the university's students for a minimal fee. 
		\item [2008] Box
		\item [2010 - to date] GULTec - ITESM Linux User Group.\\
			I organized and helped organize technical workshops, which were given for free either by me, other members of the group, or by professionals. I also was part of the organization of the event called "Interunivesitario de Software Libre 2011", which took place in various universities at the same time and was transmited via Ustream. There, I gave a presentation about Open Source Compliance. I had contact with some companies and gave announcements to the group members about the vacancies available. And, I was also in charge of administering the group's server for some months.
	\end{cvlist}

	\begin{cvlist}{Webpages \& Code Repositories}
		\item [Github] \href{http://github.com/JAORMX}{http://github.com/JAORMX}
	\end{cvlist}
	
%	\begin{cvlist}{References}
%		\item [Juan Arturo Nolazco (Director of the ITESM's Computer Science Department)]
%			\href{mailto:juan.nolazco@gmail.com}{juan.nolazco@gmail.com}
%		\item [José Ignacio Icaza (Profesor and Researcher at ITESM)]
%			\href{mailto:josei09@gmail.com}{josei09@gmail.com}
%	\end{cvlist}
	
%\date{{\footnotesize \today}}
	
\end{cv}

%\begin{center}
%	{\footnotesize Rights Reserved}
%\end{center}

%\begin{flushright}
%	{\Huge $*$}{\tiny May The Force be with Me}
%\end{flushright}

\end{document}
