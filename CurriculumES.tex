\documentclass[spanish,10pt,letterpaper]{article}
% \usepackage[LabelsAligned]{currvita}		% versión compacta
\usepackage[TextAligned]{currvita}

\usepackage[cm]{fullpage}			% márgenes de 1"
\usepackage[absolute,overlay]{textpos}		% posicionamiento absoluto

\usepackage[utf8]{inputenc}
\usepackage[spanish,english]{babel}

\usepackage{url}
\usepackage[colorlinks,urlcolor=blue]{hyperref}

\usepackage[pdftex]{color,graphicx}

%\usepackage{newcent} % New Century Schoolbook, Avant Garde, Courier 
\usepackage{palatino} % Palatino, Helevetica, Courier 
%\usepackage{bookman}
%\usepackage{lmodern}

% begin of CV
\begin{document}
\centering
\begin{cv}{Juan Antonio Osorio Robles}

% Para insertar una imagen:
% [in]{fullpage}
%	\begin{textblock}{0}(12.5,2)
%		\includegraphics[height=2.6cm]{CV-pic-small}
%	\end{textblock}
	
	\begin{table}[h]
		\begin{tabular}{@{} l l p{0.5cm} l r}
			{\bf Información Personal}	&	&	&	{\bf Contacto}	&	\\
			Fecha de Nacimiento			&	Enero 8, 1990 (21 años de edad)	&	&
			email	&	\href{mailto:jaosorior@gmail.com}	{\tt jaosorior@gmail.com}	\\
			Estado Marital		&	soltero		&	&
			Teléfono Celular				&	{\tt +52~(1)~811.583.7910}	\\
			Nacionalidad			&	Mexicano		&	&
			%land phone			&	{\tt +00~(0)~000.000.0000}
		\end{tabular}
	\end{table}
	
	\begin{cvlist}{Habilidades Técnicas}
			\item [\textsc{Lenguajes de Programación}]
				C, C++, Java, Racket (antes Scheme), Python, SQL, Javascript, PHP, Bash scripting
			\item [\textsc{Herramientas de Desarrollo}]
				Git, Subversion, Mercurial, Vim, Eclipse, gdb
			\item [\textsc {Servicios y Tecnologías}]
				Apache, Ajax, GTK+, JQuery, JSP, MapReduce, MySQL, Wave Gadgets y Robots
			\item [\textsc {Lenguajes de Marcas}]
				HTML, XML,
				\LaTeX
		\item [\textsc {Multimedia}]
				GIMP, VLC, Audacity
	\end{cvlist}

	\begin{cvlist}{Experiencia Profesional}
		\item [Julio 2011 --]
			Proyecto de investigación en ITESM para Nic México (Monterrey, N.L., México)
			\\ \emph{Diseño de arquitectura, programación(desarrollo)}
			\\ Estoy trabajando con otros cuatro compañeros en la implementación del mecanísmo NAT64 para el sistema operativo Linux, cuyo propósito es habilitar la comunicación entre clientes IPv6 e IPv4. Está siendo desarrollado como un módulo del kernel que trabaja con Netfilter. Para control en ``user-space'', se está desarrollando en paralelo un módulo de Iptables. La implementación será de código abierto y será utilizada en los servidores de NIC México.
		\item [Mayo 2010 - Mayo 2011]
			Asistente de Investigación en ITESM (Monterrey, N.L.)
			\\ \emph{Programación(Desarrollo), infraestructura, interfáz de usuario y pruebas}
			\\ Desarrollé una aplicación (Gadget) para la plataforma Google Wave que sirve para hacer diagramas ER, utilizando HTML5 y los API's the Wave. Además contribuí a la investigación documental del proyecto, así como detalles de la implementación de los robots, y los estados que éstos compartirían con el Gadget. Posteriormente contribuí con parches al proyecto Open Source Wave In A Box, estudié la infraestructura y adapté detalles del código para fines del proyecto. Por otro lado, mantuve junto con un compañero un servidor dentro del Campus para cuestiones de pruebas.
		\item [Verano 2010]
			% Organization (location):
			Sistema de información administrativo para Departamentos Coahuila (Monterrey, N.L.)
			% Position:
			\\ \emph{Programación(Desarrollo), Diseño y seguridad}
			% Responsibilities:
			\\ Un compañero y yo hicímos un sistema de información para los Departamentos Coahuila, con la cual se efectúan actividades administrativas como la impresión y registro de contratos, recibos, facturas, y control de inquilínos.
			
	\end{cvlist}

	\begin{cvlist}{Educación}
%	\begin{cvlist}{Historia académica}
		\item	[08/2008--]
				ITESM, Campus Monterrey
				\\
				\emph{BCT - Ingeniero en Tecnologías Computacionales}
				\\
				MI - Modalidad Internacional (\href{http://www.mty.itesm.mx/rectoria/dda/dacmi}{http://www.mty.itesm.mx/rectoria/dda/dacmi})
				\\
				Promedio Acumulado en la Carrera: \emph{91}
	\end{cvlist}

	\begin{cvlist}{Lenguajes}
		\item [\textsc {Español}]
				Nativo
		\item [\textsc {Inglés}]
				TOEFL: \emph{103/120}
		\item [\textsc {Alemán}]
				Intermedio, 2 años
	\end{cvlist}


	\begin{cvlist}{Intereses}
		\item [\textsc {Tecnología}]
			Cómputación distribuida, Programación en Linux y su infraestructura, infraestructura de sistemas operativos, Software Libre y Código Abierto
		\item [\textsc {Ciencias Computacionales}]
			Sistemas inteligentes, aprendizaje maquinal, lenguajes de programación
		\item [\textsc {Ciencia y Artes}]
			Filosofía, historia, literatura, música, artes gráficos
	\end{cvlist}

	% Other traits, other matters of note.
	\begin{cvlist}{Información Adicional}
		\item [  ] Usario de Linux desde 2008
		\item [  ] Promotor del "Open Source"
		\item [\textsc {Hobbies}]
				Tocar guitarra, literatura, cinematografía
	\end{cvlist}

	\begin{cvlist}{Actividades Extracurriculares}
		\item [2008--] SAITC - Sociedad de Alumnos de la carrera de Ingeniería en Tecnologías Computacionales
		\item [2008] Box
		\item [2010--] GULTec - Grupo de usuarios de Linux
	\end{cvlist}
	
	\cvplace{{\footnotesize Monterrey NL MX}}
	\date{{\footnotesize \today}}
	
\end{cv}

\begin{center}
	{\footnotesize Derechos Reservados}
\end{center}

%\begin{flushright}
%	{\Huge $*$}{\tiny May The Force be with Me}
%\end{flushright}

\end{document}
