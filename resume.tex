% LaTeX file for resume 
% This file uses the resume document class (res.cls)

\documentclass{res} 
%\usepackage{helvetica} % uses helvetica postscript font (download helvetica.sty)
%\usepackage{newcent}   % uses new century schoolbook postscript font 
\newsectionwidth{0pt}  % So the text is not indented under section headings
\setlength{\textheight}{10.2in} % set text height big enough for box
\topmargin=-.5in       % to start box .5in from top of page
\oddsidemargin=-.5in   % to start box .5in from left of page
    
\begin{document}
 
%%%%%%%%%%%%%%%%%%%%%%%%%%%%%%%%%%%%%%%%%%%%%%%%%%%%%%%%%%%%%%%%%%%%%%%%%%%%
% The following lines define \boxaround, used to draw a box on the page.
% The parameter is the entire text of the resume. Must fit on one page!
%
% \boxaroundhmargin is the left & right margin around the text inside the box.
% \boxaroundvmargin is the top & bottom margin around the text inside the box.
% \boxrulethickness controls thickness of line used to draw the box.
% You can change these 3 things in the lines below:
%%%%%%%%%%%%%%%%%%%%%%%%%%%%%%%%%%%%%%%%%%%%%%%%%%%%%%%%%%%%%%%%%%%%%%%%%%%%%
%\newdimen\boxrulethickness\newdimen\boxaroundhmargin\newdimen\boxaroundvmargin
%\boxrulethickness=.5pt        %controls thickness of line 
%\boxaroundhmargin=35pt        % about a half inch
%\boxaroundvmargin=40pt        % to fit more text on page, make this smaller
%%%%%%%%%%%%%%%%%%%%%%%%% Don't read this stuff %%%%%%%%%%%%%%%%%%%%%%%%%%%%%%
%\hsize=7.5in% \vsize=10.5in             % use bigger dimensions for box
%\newbox\MACboxA  \newdimen\MACdimenA
% \borderandboxit is used inside \boxaround:
%\def\borderandboxit#1#2#3{\vbox{\hrule height#2\hbox{\vrule width#2\hskip#1\hskip-#2%
%  \vbox{\vskip#1\relax#3\vskip#1}\hskip#1\hskip-#2\vrule width#2}\hrule height#2}}
%
%\long\def\boxaround#1{\vskip6pt
%  {\MACdimenA=\hsize \advance\MACdimenA by-\boxaroundhmargin
%   \advance\MACdimenA by-\boxaroundhmargin   % once for each side
%   \setbox\MACboxA=\hbox to \hsize{\hskip\boxaroundhmargin%\hss
%                     \vbox{\hsize=\MACdimenA
%                           \vskip\boxaroundvmargin #1
%                           \vskip\boxaroundvmargin}\hss}%
%   \borderandboxit{0pt}\boxrulethickness{\box\MACboxA}}%
%  \vskip2pt plus0pt minus0pt
%}
%%%%%%%%%%%%%%%%%%%  End of \boxaround macro %%%%%%%%%%%%%%%%%%%%%%%%%%%%%%%%%
 
%\boxaround{ % put the text on the page inside a box  

\name{Juan Antonio Osorio-Robles\\[12pt]}
\address{\bf Current Address:\\ J\"{a}mer\"{a}ntaival 5A 124, 02150 Espoo, Finland\\
Mobile: {\tt (+358)0458507886}} 
\address{\bf Permanent Address:\\ Guayanas 1021, Col. Guadalupe, \\ Monclova, Coahuila, M\'{e}xico}

 
\begin{resume}

\section{\sl  Education}
\textbf{\emph{Aalto University}}, School of Science. Helsinki, Finland. Exchange student in the Computer Science Department. January 2012 to December 2012. \\
\textbf{\emph{ITESM}}, Campus Monterrey. Monterrey, N.L., M\'{e}xico. B.S. in Computer Science and Technologies. International Modality. GPA: 3.7. August 2008 -- May 2013
 
\section{\sl Technical Skills}
 \textbf{\emph{Programming \& Markup Languages}:}
C(3 years), Java(3 years), Scheme(1.5 years), Python(6 months), SQL(2 years), Javascript(2 years), PHP(2.5 years), bash scripting (2 years), HTML, XML,\LaTeX \\
 \textbf{\emph{Development Tools, Services \& Technologies}:}
Git, Subversion, Mercurial, Vim, Eclipse, gdb, 
Apache, Ajax, GTK+, JQuery, Netfilter, MySQL, Wave Gadgets \& Robots \\
 \textbf{\emph{SW Development Methods}:}
Agile Methodologies (Extreme Programming, Scrum), with Object Oriented Design (UML) \\
\textbf{\emph{Operating Systems}:}
Linux (good and enthusiast) \\

\section{\sl Professional Experience}
\begin{ncolumn}{2}
{\it Research project in ITESM for NIC Mexico}  &   July 2011 - December 2011 
\end{ncolumn}\\
Monterrey, N.L., M\'{e}xico \\
Architect and developer lead of a Linux implementation for the NAT64 mechanism (based on the RFC6146), which had the purpose of enabling communication between IPv6 and IPv4 hosts. It was implemented as a kernel module, written in C, working with Netfilter, which would be the kernel-space component. For user-space control, an iptables module was also being developed. The implementation is open source under the GPLv3 license.

\begin{ncolumn}{2}
{\it Research project in ITESM}  &   May 2010 - May 2011 
\end{ncolumn}\\
Monterrey, N.L., M\'{e}xico \\
I developed an application (Gadget) for the Google Wave platform to collaboratively draw ER diagrams. The project used web technologies such as HTML5 and the Wave platform's API. Also, I contributed to the research document and designed the states (information) that the robots would be sharing and updating with the Gadget. I also contributed to the Open Source project Wave In a Box, studied its infrastructure and adapted code to fit the University's project. I was responsable for administrating an Ubuntu Server with a co-worker inside Campus for the project's testing purposes.

\begin{ncolumn}{2}
{\it Freelance web development project}  &  Summer 2010 
\end{ncolumn}\\
Monterrey, N.L., M\'{e}xico \\
A co-worker and I developed an information system to administer an apartment complex. It managed the tentants' contracts and payments, the apartments' availability and the printing of contracts, receipts and bills.
	
\section{\sl  Awards \& Certifications}
Second place in the ITESM's ACM Hackathon. \\
Winner (with a team of 5) of the ITESM's Bussiness Model Presentation fair. \\
Winner (with a team of 5) of the ITESM's Bussiness Model Challenge. \\
90.2/100 in the Introduction to Artificial Intelligence class (ai-class), given in partnership with Stanford Engineering
 
\section{\sl  Activities \& Interests}
ITESM's Linux User Group\\
Open Source enthusiast \\
Technological interests: Distributed computing, Linux programming and internals, operating systems internals, Intelligent systems, machine learning, programming languages \\
Other interests include: Guitar, philosophy, music, literature, films

\end{resume}

%\vfill} %    end the material being boxed.
\end{document}


